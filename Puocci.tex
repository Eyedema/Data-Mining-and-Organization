% Options for packages loaded elsewhere
\PassOptionsToPackage{unicode}{hyperref}
\PassOptionsToPackage{hyphens}{url}
%
\documentclass[
]{article}
\usepackage{lmodern}
\usepackage{amssymb,amsmath}
\usepackage{ifxetex,ifluatex}
\ifnum 0\ifxetex 1\fi\ifluatex 1\fi=0 % if pdftex
  \usepackage[T1]{fontenc}
  \usepackage[utf8]{inputenc}
  \usepackage{textcomp} % provide euro and other symbols
\else % if luatex or xetex
  \usepackage{unicode-math}
  \defaultfontfeatures{Scale=MatchLowercase}
  \defaultfontfeatures[\rmfamily]{Ligatures=TeX,Scale=1}
\fi
% Use upquote if available, for straight quotes in verbatim environments
\IfFileExists{upquote.sty}{\usepackage{upquote}}{}
\IfFileExists{microtype.sty}{% use microtype if available
  \usepackage[]{microtype}
  \UseMicrotypeSet[protrusion]{basicmath} % disable protrusion for tt fonts
}{}
\makeatletter
\@ifundefined{KOMAClassName}{% if non-KOMA class
  \IfFileExists{parskip.sty}{%
    \usepackage{parskip}
  }{% else
    \setlength{\parindent}{0pt}
    \setlength{\parskip}{6pt plus 2pt minus 1pt}}
}{% if KOMA class
  \KOMAoptions{parskip=half}}
\makeatother
\usepackage{xcolor}
\IfFileExists{xurl.sty}{\usepackage{xurl}}{} % add URL line breaks if available
\IfFileExists{bookmark.sty}{\usepackage{bookmark}}{\usepackage{hyperref}}
\hypersetup{
  pdftitle={Progetto - Data Mining and Organization},
  pdfauthor={Ubaldo Puocci},
  hidelinks,
  pdfcreator={LaTeX via pandoc}}
\urlstyle{same} % disable monospaced font for URLs
\usepackage[margin=1in]{geometry}
\usepackage{color}
\usepackage{fancyvrb}
\newcommand{\VerbBar}{|}
\newcommand{\VERB}{\Verb[commandchars=\\\{\}]}
\DefineVerbatimEnvironment{Highlighting}{Verbatim}{commandchars=\\\{\}}
% Add ',fontsize=\small' for more characters per line
\usepackage{framed}
\definecolor{shadecolor}{RGB}{248,248,248}
\newenvironment{Shaded}{\begin{snugshade}}{\end{snugshade}}
\newcommand{\AlertTok}[1]{\textcolor[rgb]{0.94,0.16,0.16}{#1}}
\newcommand{\AnnotationTok}[1]{\textcolor[rgb]{0.56,0.35,0.01}{\textbf{\textit{#1}}}}
\newcommand{\AttributeTok}[1]{\textcolor[rgb]{0.77,0.63,0.00}{#1}}
\newcommand{\BaseNTok}[1]{\textcolor[rgb]{0.00,0.00,0.81}{#1}}
\newcommand{\BuiltInTok}[1]{#1}
\newcommand{\CharTok}[1]{\textcolor[rgb]{0.31,0.60,0.02}{#1}}
\newcommand{\CommentTok}[1]{\textcolor[rgb]{0.56,0.35,0.01}{\textit{#1}}}
\newcommand{\CommentVarTok}[1]{\textcolor[rgb]{0.56,0.35,0.01}{\textbf{\textit{#1}}}}
\newcommand{\ConstantTok}[1]{\textcolor[rgb]{0.00,0.00,0.00}{#1}}
\newcommand{\ControlFlowTok}[1]{\textcolor[rgb]{0.13,0.29,0.53}{\textbf{#1}}}
\newcommand{\DataTypeTok}[1]{\textcolor[rgb]{0.13,0.29,0.53}{#1}}
\newcommand{\DecValTok}[1]{\textcolor[rgb]{0.00,0.00,0.81}{#1}}
\newcommand{\DocumentationTok}[1]{\textcolor[rgb]{0.56,0.35,0.01}{\textbf{\textit{#1}}}}
\newcommand{\ErrorTok}[1]{\textcolor[rgb]{0.64,0.00,0.00}{\textbf{#1}}}
\newcommand{\ExtensionTok}[1]{#1}
\newcommand{\FloatTok}[1]{\textcolor[rgb]{0.00,0.00,0.81}{#1}}
\newcommand{\FunctionTok}[1]{\textcolor[rgb]{0.00,0.00,0.00}{#1}}
\newcommand{\ImportTok}[1]{#1}
\newcommand{\InformationTok}[1]{\textcolor[rgb]{0.56,0.35,0.01}{\textbf{\textit{#1}}}}
\newcommand{\KeywordTok}[1]{\textcolor[rgb]{0.13,0.29,0.53}{\textbf{#1}}}
\newcommand{\NormalTok}[1]{#1}
\newcommand{\OperatorTok}[1]{\textcolor[rgb]{0.81,0.36,0.00}{\textbf{#1}}}
\newcommand{\OtherTok}[1]{\textcolor[rgb]{0.56,0.35,0.01}{#1}}
\newcommand{\PreprocessorTok}[1]{\textcolor[rgb]{0.56,0.35,0.01}{\textit{#1}}}
\newcommand{\RegionMarkerTok}[1]{#1}
\newcommand{\SpecialCharTok}[1]{\textcolor[rgb]{0.00,0.00,0.00}{#1}}
\newcommand{\SpecialStringTok}[1]{\textcolor[rgb]{0.31,0.60,0.02}{#1}}
\newcommand{\StringTok}[1]{\textcolor[rgb]{0.31,0.60,0.02}{#1}}
\newcommand{\VariableTok}[1]{\textcolor[rgb]{0.00,0.00,0.00}{#1}}
\newcommand{\VerbatimStringTok}[1]{\textcolor[rgb]{0.31,0.60,0.02}{#1}}
\newcommand{\WarningTok}[1]{\textcolor[rgb]{0.56,0.35,0.01}{\textbf{\textit{#1}}}}
\usepackage{graphicx,grffile}
\makeatletter
\def\maxwidth{\ifdim\Gin@nat@width>\linewidth\linewidth\else\Gin@nat@width\fi}
\def\maxheight{\ifdim\Gin@nat@height>\textheight\textheight\else\Gin@nat@height\fi}
\makeatother
% Scale images if necessary, so that they will not overflow the page
% margins by default, and it is still possible to overwrite the defaults
% using explicit options in \includegraphics[width, height, ...]{}
\setkeys{Gin}{width=\maxwidth,height=\maxheight,keepaspectratio}
% Set default figure placement to htbp
\makeatletter
\def\fps@figure{htbp}
\makeatother
\setlength{\emergencystretch}{3em} % prevent overfull lines
\providecommand{\tightlist}{%
  \setlength{\itemsep}{0pt}\setlength{\parskip}{0pt}}
\setcounter{secnumdepth}{-\maxdimen} % remove section numbering

\title{Progetto - Data Mining and Organization}
\author{Ubaldo Puocci}
\date{3/17/2020}

\begin{document}
\maketitle

\hypertarget{introduzione}{%
\subsection{Introduzione}\label{introduzione}}

Il progetto da me scelto comprende il dataset ``Coorti 2010- 2016
studenti di tre CdS Scuola SMFN - produttivita I anno + esame di
matematica'' e prevede l'applicazione di algoritmi di clustering per
analizzare i dati proposti.

Il dataset si presenta in questo modo:

\begin{Shaded}
\begin{Highlighting}[]
\NormalTok{dataset =}\StringTok{ }\KeywordTok{read.csv}\NormalTok{(}\StringTok{"./Data/dataset.csv"}\NormalTok{)}
\KeywordTok{summary}\NormalTok{(dataset)}
\end{Highlighting}
\end{Shaded}

\begin{verbatim}
##    CdS          Coorte     Genere    Voto_test     Crediti_convoto
##  CdS1:435   Min.   :2010   F:380   Min.   :-1.00   Min.   : 0.00  
##  CdS2:458   1st Qu.:2011   M:852   1st Qu.:12.00   1st Qu.:15.00  
##  CdS3:339   Median :2013           Median :16.62   Median :33.00  
##             Mean   :2013           Mean   :16.09   Mean   :30.75  
##             3rd Qu.:2015           3rd Qu.:21.00   3rd Qu.:48.00  
##             Max.   :2016           Max.   :25.00   Max.   :69.00  
##                                                                   
##  Crediti_totali    Voto_medio   Scuola_provenienza        Esame_Matematica
##  Min.   :  3.0   Min.   : 0.0   LS     :698                       :261    
##  1st Qu.: 18.0   1st Qu.:22.0   IT     :246        EsameMatematica:970    
##  Median : 36.0   Median :25.0   LC     : 91        MATEMATICA I   :  1    
##  Mean   : 34.1   Mean   :22.1   TC     : 67                               
##  3rd Qu.: 51.0   3rd Qu.:27.0   XX     : 42                               
##  Max.   :123.0   Max.   :31.0   AL     : 32                               
##                                 (Other): 56                               
##  Voto_Matematica Crediti_Matematica
##  Min.   : 0.00   Min.   : 1.0      
##  1st Qu.:23.00   1st Qu.:12.0      
##  Median :25.00   Median :12.0      
##  Mean   :25.04   Mean   :12.8      
##  3rd Qu.:28.00   3rd Qu.:15.0      
##  Max.   :31.00   Max.   :15.0      
##  NA's   :261     NA's   :261
\end{verbatim}

Per ogni studente abbiamo quindi le seguenti informazioni:

\begin{itemize}
\tightlist
\item
  Corso di Laurea, con 3 possibili opzioni
\item
  Coorte di iscrizione, dal 2010 al 2016 compresi
\item
  Il genere
\item
  Il voto del test d'ingresso obbligatorio per gli studenti iscritto
  alla scuola di SMFN.
\item
  Crediti che corrispondono ad esami con attribuzione di voto
\item
  Crediti che corrispondono ad esami con o senza attribuzione di voto
\item
  Il voto medio che lo studente ha ottenuto negli esami da lui superati
\item
  La scuola di provenienza prima dell'iscrizione all'Università
\item
  Se lo studente ha superato o meno l'esame di Analisi I o Matematica I
  al primo anno
\item
  Il voto conseguito al suddetto esame
\item
  Il numero di crediti conseguiti con il superamento dello stesso esame
\end{itemize}

\hypertarget{preprocessing-con-r}{%
\subsection{Preprocessing con R}\label{preprocessing-con-r}}

Prima di poter applicare i classici algoritmi di clustering, è
necessario preparari i dati per modificarne alcune caratteristiche senza
alterare od eliminare nessuna informazione contenuta nel dataset.

La colonna \texttt{Scuola\_provenienza} presenta valori ricondicibili
alla seguente legenda:

\begin{itemize}
\tightlist
\item
  LS = Liceo Scientifico
\item
  LC = Liceo Classico
\item
  IT = Istituto Tecnico Industriale
\item
  TC = Istituto Tecnico Commerciale
\item
  IP = Istituto Professionale
\item
  AL, IA, IPC, LL, XX, o cella vuota = Altro
\end{itemize}

ed è quindi necessario modificare il dato per far sì che questo sia
rappresentato nel dataset:

\begin{Shaded}
\begin{Highlighting}[]
\KeywordTok{summary}\NormalTok{(dataset}\OperatorTok{$}\NormalTok{Scuola_provenienza)}
\end{Highlighting}
\end{Shaded}

\begin{verbatim}
##      AL  IA  IP IPC  IT  LC  LL  LS  TC  XX 
##  23  32   4  12   1 246  91  16 698  67  42
\end{verbatim}

\begin{Shaded}
\begin{Highlighting}[]
\NormalTok{dataset}\OperatorTok{$}\NormalTok{Scuola_provenienza =}\StringTok{ }\KeywordTok{as.character}\NormalTok{(dataset}\OperatorTok{$}\NormalTok{Scuola_provenienza)}
\NormalTok{dataset}\OperatorTok{$}\NormalTok{Scuola_provenienza =}\StringTok{ }\KeywordTok{with}\NormalTok{(dataset,}
                                  \KeywordTok{ifelse}\NormalTok{(}
\NormalTok{                                    Scuola_provenienza }\OperatorTok\StringTok{ }\KeywordTok{c}\NormalTok{(}\StringTok{'AL'}\NormalTok{, }\StringTok{'IA'}\NormalTok{, }\StringTok{'IPC'}\NormalTok{, }\StringTok{'LL'}\NormalTok{, }\StringTok{'XX'}\NormalTok{, }\StringTok{''}\NormalTok{),}
                                    \StringTok{'Altro'}\NormalTok{,}
\NormalTok{                                    Scuola_provenienza}
\NormalTok{                                  ))}

\NormalTok{dataset}\OperatorTok{$}\NormalTok{Scuola_provenienza =}\StringTok{ }\KeywordTok{as.factor}\NormalTok{(dataset}\OperatorTok{$}\NormalTok{Scuola_provenienza)}
\KeywordTok{summary}\NormalTok{(dataset}\OperatorTok{$}\NormalTok{Scuola_provenienza)}
\end{Highlighting}
\end{Shaded}

\begin{verbatim}
## Altro    IP    IT    LC    LS    TC 
##   118    12   246    91   698    67
\end{verbatim}

La prossima colonna da analizzare è \texttt{Esame\_matematica}.

\begin{Shaded}
\begin{Highlighting}[]
\KeywordTok{summary}\NormalTok{(dataset}\OperatorTok{$}\NormalTok{Esame_Matematica)}
\end{Highlighting}
\end{Shaded}

\begin{verbatim}
##                 EsameMatematica    MATEMATICA I 
##             261             970               1
\end{verbatim}

Questa colonna ci da un'informazione molto importante: se lo studente ha
superato o meno l'esame di profitto di Analisi I o Matematica I al primo
anno. Una cella vuota sta a significare che lo studente non ha superato
l'esame. Dobbiamo quindi modificare il dato per meglio spiegare questo
fenomeno, ignorando il nome dell'esame poiché non è di nostro interesse
al momento.

\begin{Shaded}
\begin{Highlighting}[]
\NormalTok{dataset}\OperatorTok{$}\NormalTok{Esame_Matematica =}\StringTok{ }\KeywordTok{as.character}\NormalTok{(dataset}\OperatorTok{$}\NormalTok{Esame_Matematica)}
\NormalTok{dataset}\OperatorTok{$}\NormalTok{Esame_Matematica =}\StringTok{ }\KeywordTok{with}\NormalTok{(dataset,}
                                \KeywordTok{ifelse}\NormalTok{(Esame_Matematica }\OperatorTok\StringTok{ }\NormalTok{(}\StringTok{''}\NormalTok{), }\StringTok{'Non superato'}\NormalTok{, Esame_Matematica))}
\NormalTok{dataset}\OperatorTok{$}\NormalTok{Esame_Matematica =}\StringTok{ }\KeywordTok{with}\NormalTok{(dataset,}
                                \KeywordTok{ifelse}\NormalTok{(Esame_Matematica }\OperatorTok\StringTok{ }\NormalTok{(}\StringTok{'MATEMATICA I'}\NormalTok{), }\StringTok{'EsameMatematica'}\NormalTok{, Esame_Matematica))}
\NormalTok{dataset}\OperatorTok{$}\NormalTok{Esame_Matematica =}\StringTok{ }\KeywordTok{as.factor}\NormalTok{(dataset}\OperatorTok{$}\NormalTok{Esame_Matematica)}
\KeywordTok{summary}\NormalTok{(dataset}\OperatorTok{$}\NormalTok{Esame_Matematica)}
\end{Highlighting}
\end{Shaded}

\begin{verbatim}
## EsameMatematica    Non superato 
##             971             261
\end{verbatim}

Un attributo direttamente legato al precedente è
\texttt{Voto\_Matematica}.

\begin{Shaded}
\begin{Highlighting}[]
\KeywordTok{summary}\NormalTok{(dataset}\OperatorTok{$}\NormalTok{Voto_Matematica)}
\end{Highlighting}
\end{Shaded}

\begin{verbatim}
##    Min. 1st Qu.  Median    Mean 3rd Qu.    Max.    NA's 
##    0.00   23.00   25.00   25.04   28.00   31.00     261
\end{verbatim}

Come mostrato, questo attributo presenta valori pari a zero e valori
nulli. I valori pari a zero sono interpretabili come informazione non
presente nel dataset, mentre i valori nulli corrispondono agli studenti
che non hanno superato l'esame di matematica. L'informazione mancante
non può essere esclusa, considereremo quindi la media dei voti dello
studente come valore attendibile per \texttt{Voto\_Matematica}.

\begin{Shaded}
\begin{Highlighting}[]
\NormalTok{dataset}\OperatorTok{$}\NormalTok{Voto_Matematica =}\StringTok{ }\KeywordTok{with}\NormalTok{(dataset, }\KeywordTok{ifelse}\NormalTok{(Voto_Matematica }\OperatorTok\StringTok{ }\NormalTok{(}\DecValTok{0}\NormalTok{), Voto_medio, Voto_Matematica))}
\NormalTok{dataset}\OperatorTok{$}\NormalTok{Voto_Matematica =}\StringTok{ }\KeywordTok{with}\NormalTok{(dataset, }\KeywordTok{ifelse}\NormalTok{(Esame_Matematica }\OperatorTok\StringTok{ }\NormalTok{(}\StringTok{'Non superato'}\NormalTok{), }\DecValTok{0}\NormalTok{, Voto_Matematica))}
\KeywordTok{summary}\NormalTok{(dataset}\OperatorTok{$}\NormalTok{Voto_Matematica)}
\end{Highlighting}
\end{Shaded}

\begin{verbatim}
##    Min. 1st Qu.  Median    Mean 3rd Qu.    Max. 
##    0.00   19.00   24.00   19.85   28.00   31.00
\end{verbatim}

In questo modo abbiamo mantenuto le informazioni intatte all'interno del
nostro dataset, in qualche modo inferendo quelle mancanti, e modificato
il significato di un valore dell'attributo \texttt{Voto\_Matematica}:
adesso il lo zero corrisponde agli studenti che non hanno superato
l'esame di matematica.

\includegraphics{Puocci_files/figure-latex/unnamed-chunk-7-1.pdf}

Uno degli attributi più importanti di questo dataset è
\texttt{Voto\_test} che indica il voto conseguito da uno studente per il
test di ingresso al Corso di Laurea a cui si è iscritto. L'attribuzione
del voto è cambiate durante gli anni, in particolare: negli anni
2010-2015, il test era costituito da un questionario con 25 domande e
ogni risposta corretta era valutata 1, ogni risposta sbagliata o non
data era valutata 0; il test risultava superato con un punteggio
\textgreater=12. Dal 2016, invece, il test e' costituito da un
questionario con 20 domande: ogni risposta corretta viene valutata 1,
ogni risposta sbagliata viene valutata -0.25 e ogni risposta non data 0;
il test risulta superato con punteggio \textgreater=8. E' quindi
presente, a seconda dell'anno che prendiamo in considerazione, un
diverso range di valori con attributi diversi che dovrebbero in realtà
avere lo stesso significato, come per esempio 8 e 12 per il superamento
del test. Possiamo dunque applicare una tecnica di standardizzazione che
riconduce un qualunque attributo \(v\) con media \(\mu\) e varianza
\(\sigma^2\) ad una variabile con \(\mu=0\) e \(\sigma^2=1\), ossia con
distribuzione standard Definendo \(\mu_0,\mu_1\dots,\mu_6\) come la
media dei valori dell'attributo e \(\sigma_0,\sigma_1\dots,\sigma_6\) la
sua deviazione standard rispettivamente per gli anni
\(2010, 2011, \dots, 2016\), il nuovo valore è cacolato come:

\(v'=\frac{v-\mu_i}{\sigma_i}\)

attraverso la funzione \texttt{scale}.

\begin{Shaded}
\begin{Highlighting}[]
\KeywordTok{summary}\NormalTok{(dataset}\OperatorTok{$}\NormalTok{Voto_test)}
\end{Highlighting}
\end{Shaded}

\begin{verbatim}
##    Min. 1st Qu.  Median    Mean 3rd Qu.    Max. 
##   -1.00   12.00   16.62   16.09   21.00   25.00
\end{verbatim}

\begin{Shaded}
\begin{Highlighting}[]
\KeywordTok{sd}\NormalTok{(dataset}\OperatorTok{$}\NormalTok{Voto_test)}
\end{Highlighting}
\end{Shaded}

\begin{verbatim}
## [1] 5.626392
\end{verbatim}

\includegraphics{Puocci_files/figure-latex/unnamed-chunk-9-1.pdf}

\begin{Shaded}
\begin{Highlighting}[]
\NormalTok{rescale_to_}\DecValTok{01}\NormalTok{ <-}\StringTok{ }\ControlFlowTok{function}\NormalTok{(dataset, anno) \{}
\NormalTok{  subset_data =}\StringTok{ }\KeywordTok{subset}\NormalTok{(dataset, dataset}\OperatorTok{$}\NormalTok{Coorte }\OperatorTok{==}\StringTok{ }\NormalTok{anno)}
\NormalTok{  subset_data}\OperatorTok{$}\NormalTok{Voto_test =}\StringTok{ }\KeywordTok{scale}\NormalTok{(subset_data}\OperatorTok{$}\NormalTok{Voto_test)}
  \KeywordTok{return}\NormalTok{(subset_data)}
\NormalTok{\}}
\end{Highlighting}
\end{Shaded}

\begin{Shaded}
\begin{Highlighting}[]
\KeywordTok{summary}\NormalTok{(dataset}\OperatorTok{$}\NormalTok{Voto_test)}
\end{Highlighting}
\end{Shaded}

\begin{verbatim}
##        V1          
##  Min.   :-3.99885  
##  1st Qu.:-0.74231  
##  Median : 0.06651  
##  Mean   : 0.00000  
##  3rd Qu.: 0.78585  
##  Max.   : 2.37405
\end{verbatim}

\begin{Shaded}
\begin{Highlighting}[]
\KeywordTok{sd}\NormalTok{(}\KeywordTok{subset}\NormalTok{(dataset, dataset}\OperatorTok{$}\NormalTok{Coorte }\OperatorTok{==}\StringTok{ }\DecValTok{2010}\NormalTok{)}\OperatorTok{$}\NormalTok{Voto_test)}
\end{Highlighting}
\end{Shaded}

\begin{verbatim}
## [1] 1
\end{verbatim}

\includegraphics{Puocci_files/figure-latex/unnamed-chunk-13-1.pdf}

\hypertarget{including-plots}{%
\subsection{Including Plots}\label{including-plots}}

You can also embed plots, for example:

\includegraphics{Puocci_files/figure-latex/pressure-1.pdf}

Note that the \texttt{echo\ =\ FALSE} parameter was added to the code
chunk to prevent printing of the R code that generated the plot.

\end{document}
